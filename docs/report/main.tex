\documentclass{article}

\usepackage{tikz, hyperref, cleveref, apacite}

\crefformat{section}{\S#2#1#3}
\crefformat{subsection}{\S#2#1#3}
\crefformat{subsubsection}{\S#2#1#3}
\crefformat{figure}{#2Figura~#1#3}
\crefformat{footnote}{#2\footnotemark[#1]#3}
\crefformat{table}{#2Tabella~#1#3}

\usepackage[bottom]{footmisc}
\usepackage[margin=1in]{geometry}
\renewcommand{\contentsname}{Indice}
\renewcommand{\refname}{Riferimenti}


\title{RT-Jam}

\author{Giuseppe Capasso}

\begin{document}

\begin{titlepage}
  \thispagestyle{empty}
  \raggedright % Allinea a sinistra

  \begin{tikzpicture}
    \node[anchor=south west] at (4,0) {\includegraphics[scale=0.75]{./figures/unina-logo-1.png}};
    \node[anchor=south west] at (0,1.5) {\includegraphics{./figures/unina-logo-2.png}};
    \node[anchor=south west] at (0,0.5) {\textsf{Scuola Politecnica e delle Scienze di Base}};
    \node[anchor=south west] at (0,0) {\textsf{Corso di Laurea Magistrale in Ingegneria Informatica}};
  \end{tikzpicture}

  \vfill

  {\textbf{\textit{\LARGE RT-Jam}}}
  \\[2cm]

  {\textbf{\textit{\Large Elaborato di Web and Real Time Communication Systems}}}
  \\[1cm]
  {\large Anno Accademico 2023/2024}

  \vfill

  \begin{table}[h]
    \textbf{Giuseppe Capasso}
    \\
    \textbf{matr. M63001498}
  \end{table}

\end{titlepage}

\tableofcontents
\newpage

\section*{Introduzione}
Il progetto ha come obiettivo quello di esplorare un modo più moderno di approcciarsi al 
mondo dello sviluppo web.  Oggigiorno le applicazioni devono essere altamente interattive e
vanno oltre la semplice visualizzazione di immagini statiche e pagine HTML per cui era stato
pensato il web inizialmente. Infatti, il web prende un'accezione sempre più larga
in cui alla navigazione classica mediante un browser si aggiungono requisiti di sicurezza,
performance e comunicazioni a bassa latenza. 

\clearpage
\section{Architettura}
Il 
\clearpage

\clearpage
\section{Installazione}
\clearpage


\bibliography{refs}
\bibliographystyle{apacite}
\end{document}
