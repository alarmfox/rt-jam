\documentclass{article}

\usepackage{tikz, hyperref, cleveref, apacite}

\crefformat{section}{\S#2#1#3}
\crefformat{subsection}{\S#2#1#3}
\crefformat{subsubsection}{\S#2#1#3}
\crefformat{figure}{#2Figura~#1#3}
\crefformat{footnote}{#2\footnotemark[#1]#3}
\crefformat{table}{#2Tabella~#1#3}

\usepackage[bottom]{footmisc}
\usepackage[margin=1in]{geometry}
\renewcommand{\contentsname}{Indice}
\renewcommand{\refname}{Riferimenti}


\title{RT-Jam}

\author{Giuseppe Capasso}

\begin{document}

\begin{titlepage}
  \thispagestyle{empty}
  \raggedright % Allinea a sinistra

  \begin{tikzpicture}
    \node[anchor=south west] at (4,0) {\includegraphics[scale=0.75]{./figures/unina-logo-1.png}};
    \node[anchor=south west] at (0,1.5) {\includegraphics{./figures/unina-logo-2.png}};
    \node[anchor=south west] at (0,0.5) {\textsf{Scuola Politecnica e delle Scienze di Base}};
    \node[anchor=south west] at (0,0) {\textsf{Corso di Laurea Magistrale in Ingegneria Informatica}};
  \end{tikzpicture}

  \vfill

  {\textbf{\textit{\LARGE RT-Jam}}}
  \\[2cm]

  {\textbf{\textit{\Large Elaborato di Web and Real Time Communication Systems}}}
  \\[1cm]
  {\large Anno Accademico 2023/2024}

  \vfill

  \begin{table}[h]
    \textbf{Giuseppe Capasso}
    \\
    \textbf{matr. M63001498}
  \end{table}

\end{titlepage}

\tableofcontents
\newpage

\section*{Introduzione}
Il progetto ha come obiettivo quello di esplorare un modo più moderno di approcciarsi al 
mondo dello sviluppo web.  Oggigiorno le applicazioni devono essere altamente interattive e
vanno oltre la semplice visualizzazione di immagini statiche e pagine HTML per cui era stato
pensato il web inizialmente. Infatti, il web prende un'accezione sempre più larga
in cui alla navigazione classica mediante un browser si aggiungono requisiti di sicurezza,
performance e comunicazioni a bassa latenza. 

In particolare, il progetto utilizza uno stack software basato su Rust che utilizza
\textit{WebAssembly} per interagire con il browser ed esplorare il protocollo QUIC per
realizzare un'applicazione di streaming audio a bassa latenza.
\clearpage
\section{Architettura}
\clearpage
\section{Implementazione}
\subsection{Tecnologie utilizzate}
\subsubsection{Stack software}
Il progetto è realizzato completamente in Rust: un linguaggio compilato fortemente tipizzato
con un meccanismo di gestione della memoria innovativo basato sul concetto di
\textbf{\textit{borrowing}}. 
Infatti, usando il \textbf{\textit{borrow checker}} Rust riesce a fare a meno di un 
\textit{runtime} con un \textit{garbage collector} garantendo il concetto di \textit{memory
safety} a tempo di compilazione\footnote{Con \textit{memory safety} si intende l'assenza 
di vulnerabilità del tipo \textit{use after free, double free} e accesso a puntatori
invalidi}.

\paragraph{Web framework: Axum} Axum è un framework web basato sul runtime asincrono
\textit{Tokio} che consente di effettuare multiplexing delle connessioni sfruttando le 
risorse fisiche della macchina in maniera efficiente. Come tutti i web framework, quali
Spring, ASP.net etc., Axum consente di definire un'architettura a middleware basata sui
pattern IoC (Inversion of Control) e dependency injection. 

\paragraph{Database communication: sqlx} Grazie al sistema delle \textit{macro} compilate,
è possibile utilizzare una libreria che fa affidamento su \textit{sql} per la comunicazione 
con il database. In questo modo, viene eliminato l'overhead della \textit{mapping} delle 
righe del database in oggetti.

\clearpage
\section{Installazione}
\clearpage


\bibliography{refs}
\bibliographystyle{apacite}
\end{document}
